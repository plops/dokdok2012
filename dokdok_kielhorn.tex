%% This is a template that demonstrates the submission of a contribution to DoKDoK
%%
\documentclass[a4paper, 11pt]{article}

\usepackage[utf8]{inputenc} %if your LaTeX editor does not support Unicode (UTF8), 
% change this option to ansinew or latin1 
% if you need additional packages, load them here, but mind possible interference
% with the DokDok package!
\usepackage{courier}
% now, load the generic DokDok package
\usepackage{dokdok}

\usepackage{color}

\begin{document}

\dokdoktitle{Spatio-angular microscopy}

\dokdokauthor{Martin Kielhorn\sup{*1,2}, and Rainer Heintzmann\sup{1,2,3}}

\dokdokaffiliation{ \sup{1}Institute of Photonic Technology,
  Albert-Einstein Str. 9, 07745 Jena, Germany}

\dokdokaffiliation{ \sup{2}Randall Division of Cell \& Molecular
  Biophysics, King's College London, NHH, Guy's Campus, London SE1
  1UL, U.K.}

\dokdokaffiliation{ \sup{3}Institute of Physical Chemistry,
  Friedrich-Schiller-Universität Jena, Helmholtzweg 4, 07743 Jena,
  Germany}

\dokdokemail{kielhorn.martin@gmail.com}

\begin{abstract}
  \noindent \bfseries{ Photobleaching and phototoxicity pose a problem
    in live cell imaging. Excessive excitation light can induce
    reactive oxygen species in observed organisms. These can disturb
    signalling pathways and alter the natural response of the
    sample. We augment a widefield epifluorescence microscope with two
    spatial light modulators (SLM). These allow illumination with
    arbitrary patterns. Depending on the distribution of fluorophores
    a considerable reduction in photobleaching and phototoxicity can
    be expected.}
\end{abstract}

\begin{multicols}{2}

\dokdoksection{Introduction}

In a fluorescence microscope excitation light is shone through an
objective onto a specimen containing fluorophores. These fluorophores
absorb excitation light and subsequently emit photons of lower
energy. The fluorescence light of in-focus fluorophores is formed into
a sharp image, light from out-of-focus areas deteriorates the image by
creating a blurred background.

Nowadays it is common to observe the dynamic properties of
fluorescently labelled proteins and extract quantitative information
about the structures they form within living cells. The length of the
study and the accuracy of the result of a given experiment, however,
is limited by photobleaching and light induced toxicity in the
biological specimen.

When fluorophores are optically excited, they occasionally end up in
the first excited triplet state $T_1$ by intersystem crossing. The
lifetimes of $T_1$ are sufficiently long, that the excited
fluorophores react with molecular oxygen in a triplet--triplet
annihilation reaction \cite{Linde2011a}. The resulting singulet oxygen
will react quickly with neighbouring molecules and can locally damage
the cell. This process can lead to cell death or more subtle effects
such as the suppression of normal cell signalling functions.

In order to extend experiments in time and to follow the fate of cells
over many generations, a substantial reduction of unnecessary
illumination would be desired.

In a confocal microscope the illumination light has a double-cone
shape. During acquisition of an image stack, each slice is delivered
the same dosage due to energy conservation. The \emph{2-photon
  microscope} only excites fluorophores in a point-like region
surrounding the focus \cite{Denk1990} and therefore reduces
dosage on out-of-focus fluorophores. However, the non-linear processes
in the focus can induce more severe photochemical effects in the
specimen.

In ultra microscopy \cite{Siedentopf1903,Voie1993} a thin sheet of
excitation light is sent from the side into the focal plane of a
widefield microscope. This technique allowed to create time-lapse
recordings of embryogenesis in exceptional quality \cite{Huisken2004}.
Microscope objectives with big working distance are often used. This
limits the achievable resolution. Mounting the specimen is another
difficulty of this technique.

Alternatively, using a standard high numerical aperture oil immersion
objective one can illuminate a specimen that is embedded in a medium
of lower optical density than the oil with a sheet of light at a
highly inclined angle \cite{Tokunaga2008,Konopka2008}. However, this
simple technique has a considerably smaller usable field of view.

Dunsby describes a more sophisticated technique, with an increased
field of view for oblique illumination of index-matched specimen in
\cite{Dunsby2008}.

In \cite{Hoebe2007} a variant of a confocal microscope is developed,
that illuminates the specimen according to local fluorophore
concentration. Dim areas are only illuminated until the fluorophore
content is confirmed to be below the threshold. Areas with high
fluorophore content are only illuminated until a certain photon count
is reached. Areas of intermediate brightness are illuminated for the
full time. The resulting image is of the same perceived image quality
as a conventional confocal image but at much lower dosage.

For temporal focusing \cite{Oron2005} a ultra-short pulse is
diffracted at a grating in the intermediate image plane and enters the
back focal plane of the objective as a spectrum along a meridional
line. Only in the focal plane, where all colours arrive
simultaneously, is the intensity sufficient to excite the
fluorophores. This technique has been combined with generalized phase
contrast for spatial control \cite{Papagiakoumou2010}.

Finally \cite{Levoy2009} is a microlens-based attempt to build a
microscope for simultaneous detection of angular and spatial
information about the light returning from the sample. As all
light-field techniques a trade-off between spatial and angular
resolution has to be made. This is unacceptable in many cases of high
resolution microscopy. However applying their technique for
illumination would give a microscope that is comparable to ours.

\dokdoksection{Spatio-angular microscope} SLM1 is imaged into the back
focal plane $P$ of the objective. SLM2 is conjugate to the focal plane
$F$ in sample space.  SLM1 controls the illumination angles while SLM2
selects the area in the sample, that will be illuminated.

\begin{figure}[H]
%\includegraphics[width=0.48\textwidth]{figure1.png}
%\includegraphics[width=0.48\textwidth]
{\scriptsize \input{memi-real.pdf_tex}}
\caption{Some interesting pictures. (a) Schloss Oppurg. (b) Directions.}
\label{fig:figure1}
\end{figure}

\end{multicols}

\begin{center}
\rule{0.75\textwidth}{1pt}
\end{center}

\bibliographystyle{dokdok}
\bibliography{all}


\end{document}


